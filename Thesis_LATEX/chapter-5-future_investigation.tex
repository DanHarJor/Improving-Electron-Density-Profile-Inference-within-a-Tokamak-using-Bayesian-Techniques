\chapter{Future Investigation}

% \begin{itemize}
%     \item In future the injected gas information could be included to get a prior mean vector. How does this affect the inference? Is the widely accepted 0 mean prior a good idea? Simply make the prior mean high enough so that 99\% of the area is in the positive region. Since we know that the electron density is certainly positive. 
%     \item Plolarimetry information could be included. Using a thermal profile to get current profile to get polodial magnetic field profile to then use the polarimetry measurements to get more data on the current density. Explain why NICE didn't do this and why this might allow another implementation to improve on NICE.
%     \item try maximizing likelyhood rather than marginal likelyhood for best parameters. See obsidian closely fitting the data. 
%     \item Full monte carlo if not done, to get a bayesian ground truth.
%     \item Explain potential for real time inference, why its is important and how you envision it could be done. Using data from many shots and times we could compute the least square error of inference with bayesian ground truth as the loss function to optimise a one size fits all kernel with many parameters. Each l(rho) and sigma(rho) has a different value then the kernel would only have sigma along the diagonal as l can be optimized on the off diagonals to account for different sigma values. This reduces the chance of non-positive definite matrix errors. Then using posterior of one inference as prior of next. Why is this method better than a neural network?
%     \item Jeffry finds the parameters are the same for many shots, he thinks I can take the mean and it will work fine as a one size fits all kernel.
%     \item Jeffries Advanced Static Kernel, a reference prior can be used to direct the functional form limitations of a non static kernel which is the Advanced Static Kernel.  The reference prior could be NICE, A monte carlo inference for 1 shot, or a parabolic with correct core and edge with edge being 0 and core given by amount of gas injected. It just needs to follow the rough shape as the implimentation is general. Although I fear if I use a parabolic the inferences will be parabolic. 
%     \item Jeffries Advanced Static Kernel could be used to help get a one size fits all kernel. You could learn the $sigma_{ij}$ and $l_{ij}$ that lead to the lowest loss over many shots and you don't need to worry about non-positive definite errors as $sigma_{ij}$ and $l_{ij}$ are fed into the the Advanced static kernel formula. Then you have very many parameters as 
%  \end{itemize}