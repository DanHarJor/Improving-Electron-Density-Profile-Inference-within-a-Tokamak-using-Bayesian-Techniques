\chapter{Deriving the Closed Form Posterior Expressions}
% \label{Appendix A}
\label{append:dervcf}
The inference begins with Bayes theorem,  

\begin{equation}    
    P(\vec{y}|\vec{d}, \vec\epsilon, \theta) = \frac{P(\vec{d}|\vec{y},\vec\epsilon)P(\vec{y}|\theta)}{P(\vec d|\vec\epsilon,\theta)},
\end{equation}

where the likelihood can be written as,

\begin{equation}
P(\vec{d}|\vec{y},\vec\epsilon) = \frac{1}{(2\pi)^{\frac{m}{2}} \sqrt{|\Sigma_{li}|}} \exp \left[ -\frac{1}{2} (\vec d - R\vec y)^{\top} \Sigma^{-1}_{li} (\vec d - R\vec y) \right], \, \Sigma_{li} = \vec \epsilon I,
\end{equation}

the prior as,
\begin{equation}
\begin{split}
P(\vec y|\theta) = \frac{1}{(2\pi)^{\frac{n}{2}} \sqrt{|K|}} \exp \left[ -\frac{1}{2}(\vec y - \vec \mu_{pr})^{\top} K^{-1} (\vec y - \vec \mu_{pr}) \right],\\
\theta \rightarrow \{\sigma, l\}, \, K_{ij} = k(\rho_i, \rho_j) = \sigma \exp \left[\frac{(\rho_i - \rho_j)^2}{2l^2}\right],
\end{split}
\end{equation}

and the posterior as,

\begin{equation}
P(\vec{y}|\vec{d},\vec\epsilon, \theta) = \frac{1}{(2\pi)^{\frac{n}{2}} \sqrt{|\Sigma_{post}|}} \exp \left[ -\frac{1}{2}(\vec y - \mu_{post})^{\top} \Sigma^{-1}_{post} (\vec y - \mu_{post}) \right].
\end{equation}

To derive $\mu_{post}$ and $\Sigma_{post}$ the likelihood and prior are multiplied together and re-arranged. Only first and second order $\vec y$ terms are kept as the constants do not affect the shape of the multi variate Gaussian and thus do not affect $\mu_{post}$ or $\Sigma_{post}$. Then using the completing the square formula for matrices they can be combined into a single multivariate Gaussian. By comparing with the posterior we find the closed form expressions for $\mu_{post}$ and $\Sigma_{post}$. When the distributions are multiplied together the exponential powers are summed,

$$
 -\frac{1}{2}\left[(\vec d - R\vec y)^{\top} \Sigma^{-1}_{li} (\vec d - R\vec y)  + (\vec y - \vec \mu_{pr})^{\top} K^{-1} (\vec y - \vec \mu_{pr})\right],
$$

\noindent ignoring the $-\frac{1}{2}$ for now and multiplying it out gets,

\begin{multline*}
\left(\vec d ^\top \Sigma_{li}^{-1} \vec d - \vec d^\top \Sigma_{li}^{-1} R\vec y - (R\vec y)^\top \Sigma_{li}^{-1} \vec d + (R\vec y)^\top \Sigma_{li}^{-1} R\vec y \right), \\ 
+ \left( \vec y^\top K^{-1} \vec y - \vec y^\top K^{-1} \vec \mu_{pr} - \vec \mu_{pr}^\top K^{-1} \vec y + \vec \mu_{pr}^\top K^{-1} \vec \mu_{pr} \right),
\end{multline*}

\noindent focusing on the 1st order terms and remembering that the transpose of a scalar is itself and the transpose of a symmetric matrix (e.g. $\Sigma_{li}$) is itself, it can be shown that the first order terms equate to

$$
- \vec d^\top \Sigma_{li}^{-1} R\vec y - (R\vec y)^\top \Sigma_{li}^{-1} \vec d - \vec y^\top K^{-1} \vec \mu_{pr} - \vec \mu_{pr}^\top K^{-1} \vec y = -2 \vec y^\top ( R^{\top} \Sigma_{li}^{-1}\vec d + K^{-1} \vec \mu_{pr}) =-2 \vec y^\top \vec b
$$

\noindent in which a substitution was made to ease the use of the competing square formula, 

$$
\vec b = R^{\top} \Sigma_{li}^{-1}\vec d + K^{-1} \vec \mu_{pr}
$$

\noindent switching the focus to the 2nd order terms,

$$
(R\vec y)^\top \Sigma_{li}^{-1} R\vec y + \vec y^\top K^{-1} \vec y = \vec y^\top (R^\top \Sigma_{li}^{-1} R + K^{-1}) \vec y = \vec y^\top M \vec y,
$$

\noindent in which a substitution was made to ease the use of the completing square formula,

$$
M = (R^\top \Sigma_{li}^{-1} R + K^{-1})
$$

\noindent ignoring 0 order terms that do not affect the shape, the original exponential power takes the form,

$$
 -\frac{1}{2}\left[\vec y^\top M \vec y - \vec y^\top \vec b \right],
$$

\noindent by completing the squares we obtain
$$
\vec y^\top M \vec y - y^\top \vec b = (\vec y - M^{-1}\vec b)^\top M (\vec y - M^{-1} \vec b) - \vec b^\top M^{-1} \vec b.
$$

\noindent We can ignore $\vec b^\top M^{-1} \vec b$ as it doesn't affect the shape of the Gaussian. Finally, for the posterior we have

$$
P(\vec{y}|\vec{d},\vec\epsilon, \theta) \propto \exp \left[ -\frac{1}{2}(\vec y - \mu_{post})^{\top} \Sigma^{-1}_{post} (\vec y - \mu_{post}) \right] \propto \exp \left[ -\frac{1}{2} (\vec y - M^{-1}\vec b)^\top M (\vec y - M^{-1} \vec b)\right],
$$

\noindent from comparison, it can be seen that,

\begin{equation}
\mu_{post} = M^{-1} \vec b = \left(R^\top \Sigma_{li}^{-1} R + K^{-1}\right)^{-1} \left(R^{\top} \Sigma_{li}^{-1}\vec d + K^{-1} \vec \mu_{pr}\right), \, \Sigma_{post} = M^{-1} = \left(R^\top \Sigma_{li}^{-1} R + K^{-1}\right)^{-1}.
\end{equation}

The posterior mean is often written in another form. This form can be found with the following steps,

$$
\begin{aligned}
\vec{\mu}_{post} &= (K^{-1} + R^{\top} \Sigma_{li}^{-1} R)^{-1}(R^{\top} \Sigma_{li}^{-1} \vec{d} + K^{-1} \vec{\mu}_{pr}) \\
&= (K^{-1} + R^{\top} \Sigma_{li}^{-1} R)^{-1} R^{\top} \Sigma_{li}^{-1} \vec{d} + (K^{-1} + R^{\top} \Sigma_{li}^{-1} R)^{-1} (K^{-1} + R^{\top} \Sigma_{li}^{-1} R - R^{\top} \Sigma_{li}^{-1} R) \vec{\mu}_{pr} \\
&= \vec{\mu}_{pr} + (K^{-1} + R^{\top} \Sigma_{li}^{-1} R)^{-1} R^{\top} \Sigma_{li}^{-1} \vec{d} - (K^{-1} + R^{\top} \Sigma_{li}^{-1} R)^{-1} R^{\top} \Sigma_{li}^{-1} R \vec{\mu}_{pr} \\
&= \vec{\mu}_{pr} + (K^{-1} + R^{\top} \Sigma_{li}^{-1} R)^{-1} R^{\top} \Sigma_{li}^{-1} (\vec{d} - R \vec{\mu}_{pr}).
\end{aligned}
$$

\noindent The final closed form expression of the posterior mean and covariance is

\begin{gather}
\mu_{post}= \vec{\mu}_{pr} + (K^{-1} + R^{\top} \Sigma_{li}^{-1} R)^{-1} R^{\top} \Sigma_{li}^{-1} (\vec{d} - R \vec{\mu}_{pr})\\
\Sigma_{post} = \left(R^\top \Sigma_{li}^{-1} R + K^{-1}\right)^{-1}.
\end{gather}

\noindent The error of each value in $\mu_{post}$ can be found on the diagonal of $\Sigma_{post}$.
