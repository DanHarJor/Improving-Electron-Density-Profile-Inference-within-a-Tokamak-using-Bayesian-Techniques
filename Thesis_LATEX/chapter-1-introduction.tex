%======================================================================
\chapter{Introduction}
%======================================================================
\begin{itemize}
    \item introduce magnetic confinement fusion, the tokamak and the electron density profile
    \item Justify topic, why the electron density profile is important, why interferometry is a good diagnostic, why Bayesian inference is good in particular \gls{gpr}. How this can be expanded to combine other diagnostics in an bayesian integrated analysis. 
        \subitem{Quality of confinement ie distinguishes between H and L mode}
        \subitem{Key role in ignition criteria, explain section from Wesson Tokamaks}
        \subitem{Operation limits, Low density electrons can run away, high density over greenwald limit leads to instabilities. I imagine H-mode reduces runaway electrons}
    \item give the details of the WEST tokamak and the IMAS database system. 
    \item explain that NICE exists and that it will be the main source of comparison
    \item Summarise all work in a few paragraphs
    \item Give a literature review on what is the current status of the research in this topic.
        \subitem{Cholinskey Thompson Scattering implementation of \gls{gpr}.}
        \subitem{Show rough number on papers using GPR in fusion, perhaps a history of it.}
        \subitem{What other diagnostics can be used, how many have papers that use a \Gls{gpr} analysis?}
        \subitem{What are the other inference algorithms for accomplishing the same goal. I imagine for each diagnostic the algorithm is very different whilst bayesian methods are more widly applicable, GPR can be used for every diagnostic with a linear forward model and monte carlo sampling techniques can be used for all of them, thus simplifying the field.}
    \item Describe what is in the other sections of the thesis
\end{itemize}



