%======================================================================
\chapter{Introduction}
%======================================================================
Nuclear fusion, the process by which two light atomic nuclei combine to form a heavier nucleus, holds the promise of revolutionizing the world's energy landscape. With the ever-increasing demand for sustainable, clean, and virtually limitless energy sources, nuclear fusion has emerged as a leading candidate. Nuclear fusion, in contrast to the fission processes used in current nuclear power plants, presents a fundamentally safer and more sustainable option. The primary fuel for fusion, isotopes of hydrogen, is abundant and can be extracted from water sources. Fusion reactions produce no long-lived, highly radioactive waste, minimizing environmental hazards and long-term disposal issues. Recent advancements have reignited interest in the field, with notable breakthroughs such as the development of high-temperature superconducting magnets, which enable the efficient confinement of high-energy plasma. An excelent example is the rare earth barium copper oxide high-temperature superconducting magnets that are being deployed in the SPARC experimental reactor \cite{SPARCoverview}. In February 2022, the UK-based JET laboratory reported that it had smashed its own world record for the amount of energy it could extract by squeezing together two forms of hydrogen, deuterium and tritium. The experiments produced 59 megajoules of energy over five seconds (11 megawatts of power), which was more than double what was achieved in similar tests back in 1997. Another breakthrough was announced in December 2022 by US scientists at the National Ignition Facility in California. They confirmed that they had achieved ignition for the first time, by firing up to 192 giant lasers into a peppercorn-sized fuel pellet and triggering a fusion reaction that released more energy than was put in by the lasers \cite{nationalignitionfacilitycali}.

Tokamaks are a class of fusion device with a great potential to achieve commercial fusion energy production. Tokamaks use magnetic fields to confine and heat plasma, a state of matter where atoms are split into electrons and nuclei. The immence pressures and temperatures created induce fusion reactions between light nuclei, such as hydrogen, to produce energy. Tokamaks currently demonste long plasma confinement times, which measure how well the plasma is isolated from the surrounding environment and affect the efficiency of energy production and heat loss. Tokamaks have the most extensive scientific and technological knowledge base, which has been accumulated over decades of research and development. This leads to reliable and robust designs, advanced diagnostics and control systems, and proven solutions for engineering challenges. These advantages make tokamaks the most promising candidates for achieving commercial fusion energy in the near future. Tokamaks have already shown impressive results in terms of fusion power output and energy gain. They are expected to reach even higher levels of performance with the next generation of devices. Tokamaks are also supported by a strong international collaboration and a clear roadmap for development. Therefore, tokamaks have a huge potential to commercialise fusion before other methods: especially with ITER just around the corner and plans for DEMO already underway. ITER and DEMO are complementary projects that will advance fusion energy from the experimental stage to the commercial stage. ITER will provide the scientific and technological basis for DEMO, which will be the first fusion power plant to produce electricity and operate with a closed fuel cycle. The construction of ITER is expected to be completed by 2025, and the first plasma operation is planned for 2026. The full deuterium-tritium operation of ITER is scheduled for 2035, which will coincide with the start of the construction of DEMO. The operation of DEMO is foreseen to begin in the 2040s, and to demonstrate the viability of fusion energy for commercial use.

The electron density profile is a key parameter that affects the performance and stability of tokamak plasmas. It affects the plasma current, the confinement time, the energy transport, the magnetohydrodynamic (MHD) modes, and the coupling of external heating and current drive sources. There are physical limits that constrain the maximum achievable density in tokamak plasmas. One of the most well-known density limits is the Greenwald limit, which states that the line-averaged density cannot exceed a value proportional to the plasma current divided by the plasma cross-sectional area. This limit is empirically observed in many tokamaks, and attempts to exceed it result in disruptions or edge localized modes (ELMs). The physical mechanism behind the Greenwald limit is not fully understood, but it may be related to the stability of the edge pedestal, the bootstrap current, or the core particle transport. A lower electron density leads to runaway electrons. The magnetic fields accelerate the electrons continously and only colisions prevent them from gaining enough energy to escape the magnetic confinement. A low density can mean that collisions are not frequent enough to prevent escape and the electrons can cause serious damage to the plasma facing wall. Thus it is not benificial for the plasmas electron density profile to be such as to have many electrons with a low density near the edge. 

One of the main challenges in measuring the electron density profile is to obtain high spatial and temporal resolution over a wide radial range. Several diagnostic techniques have been developed and applied to tokamaks, such as interferometry, reflectometry, Thomson scattering, and spectroscopy. Each technique has its own advantages and limitations in terms of accuracy, reliability, coverage, and invasiveness. A combination of different techniques is often used to obtain a comprehensive picture of the electron density profile evolution.

Another worthy challenge is to maintain the electron density profile in a desired shape. The electron density profile is influenced by various factors, such as plasma geometry/magnetic configuration, plasma current and pressure, impurity content, fueling and pumping methods, and external heating and current drive sources. Some of these factors can be manipulated by the operators, others can be controlled in real time with sophisticated algorythems and feedback loops; achieving a favorable electron density profile that enhances the plasma performance and stability.

The electron density profile is an important parameter that determines many aspects of tokamak plasmas. Measuring and controlling the electron density profile is a crucial task for optimizing the tokamak operation and achieving fusion energy goals.

This thesis focuses on performing a Bayesian inference of the electron density profile using the interferometry diagnostic and a Gaussian process prior. Bayesian inference with a Gaussian process prior is a powerful technique for nonparametric modeling of complex and nonlinear phenomena. A Gaussian process is a collection of random variables, of which have a joint Gaussian distribution. A Gaussian process can be specified by a mean vector and a covariance matrix, which encode the prior assumptions about the unknown profile to be learned. By applying Bayes’ rule, one can obtain the posterior distribution of the profile given some observed data, and use it for prediction and uncertainty quantification. Interferometry is a technique that uses the interference of electromagnetic waves to measure the properties of a medium. An interferometer consists of a coherent source of radiation, such as a laser, that is split into two beams: one that passes through the plasma and one that bypasses it. The two beams are then recombined and detected by a receiver. The phase difference between the two beams depends on the difference in the optical path length, which is affected by the electron density along the line of sight. By measuring the phase difference, one can calculate the line-averaged electron density of the plasma. The interferometer within a tokamk consists of many such laser beams penetrating the plasma at various angles. This thesis uses the \gls{west} tokmak's laser geometry which covers a span of the polodial cross section. Although there is not enough information to completly and accuratly reconstruct the electron density profile a best guess given the data can be infered. It is difficult or impossible to know how close any inffered profile is to the true profile.

Given the importance of the electron density profile, regular use of interferomety and power of Bayseian inference with Gaussian process priors; it comes to little supprise that these things have been brought together before. **List the papers that do the same / similar thing and say if you do anything more/different. Mention what has been done at west ie NICE. 


write about whats included in each chapter, 

then summarise the introduction




update the interferometry section in the background theory to match this one.


maybe reword somethings to sound like yourself a bit more. Tone down some of the scientific claims/ add refferences.

Add some images to make it more lively. Possibly a nice fuision of whatever fuses in west. 

Introduce the main concepts, ideas and motivation. Include a small literature review on related works, including \gls{nice} and \gls{west}. Introduce the outline of the thesis.

% \begin{itemize}
%     \item introduce magnetic confinement fusion, the tokamak and the electron density profile
%     \item Justify topic, why the electron density profile is important, why interferometry is a good diagnostic, why Bayesian inference is good in particular \gls{gpr}. How this can be expanded to combine other diagnostics in an bayesian integrated analysis. 
%         \subitem{Quality of confinement ie distinguishes between H and L mode}
%         \subitem{Key role in ignition criteria, explain section from Wesson Tokamaks}
%         \subitem{Operation limits, Low density electrons can run away, high density over greenwald limit leads to instabilities. I imagine H-mode reduces runaway electrons}
%     \item give the details of the WEST tokamak and the IMAS database system. 
%     \item explain that NICE exists and that it will be the main source of comparison
%     \item Summarise all work in a few paragraphs
%     \item Give a literature review on what is the current status of the research in this topic.
%         \subitem{Cholinskey Thompson Scattering implementation of \gls{gpr}.}
%         \subitem{Show rough number on papers using GPR in fusion, perhaps a history of it.}
%         \subitem{What other diagnostics can be used, how many have papers that use a \Gls{gpr} analysis?}
%         \subitem{What are the other inference algorithms for accomplishing the same goal. I imagine for each diagnostic the algorithm is very different whilst bayesian methods are more widly applicable, GPR can be used for every diagnostic with a linear forward model and monte carlo sampling techniques can be used for all of them, thus simplifying the field.}
%     \item Describe what is in the other sections of the thesis
% \end{itemize}



